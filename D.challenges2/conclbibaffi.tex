% \section{Conclusion}
% A closed orientable 3-manifold is denoted {\em $n$-small} if it is induced by surgery on
% a blackboard framed link with at most $n$ crossings.
% Our bet is that both pairs of 3-manifolds in the 2 first sections of 
% this short note are not homeomorphic. This would mean that the $9$-small manifolds are
% completely classified and that
% the combinatorial dynamics of Chapter 4 in \cite{lins1995gca} based 
% on $TS$-moves which leads to a  (small, in the case of hyperbolic 3-manifolds)
% number of minimal gems, named the {\em attractor of
% the 3-manifold} is successful. This induces an efficient algorithm which 
% is capable of classifying topologically all the 3-manifolds given as a blackboard framed link
% with up to (so far) 9 crossings and maintains live the two Conjectures of page 15 of \cite{lins1995gca}:
% the $TS$- and $u^n$-moves yield an efficient algorithm
% to classify $n$-small 3-manifolds by explicitly displaying homeomorphisms, whenever they exist.

\section{A concluding remark}
By the way, the elegant drawings of blinks and blackboard framed links produced by BLINK are 
possible due the groundbreaking algorithm of R. Tamasia \cite{tamassia1987egg}.
Lauro could implement the drawings very fast because we had at hand the implementation of
network flow algorithms he had done for a project to solve timetable problems.
This is an example of the unicity in Mathematics, advocated by L. Lovasz in his famous essay \cite{lovasz1998om}.
To get the drawings one has to apply three times the full strength of network flow theory.
The drawings BLINK presents are in an integer grid and 
deterministically minimize the number of $\pi/2$-bents in the blackboarded framed links.
In particular, it permit us to deal with the unavoidable curls which adjust the integer framing in
the best possible way: we do not care about them. 
The drawings for the companion blinks are a slight modification: it replaces each $p$-valent vertex $p>4$,
by a $p$-polygon inducing 3-valent ones. The final result is massaged a bit to
produce aesthetically nice and unambiguous drawings.


%-----------------------------------
\bibliographystyle{plain}
%\bibliographystyle{is-alpha}
%\addcontentsline{toc}{bibliografia}{\MakeTextUppercase{Referências Bibliográficas}}
%\bibliography{d:/slsl\3.DadosSostenes.35.ArtigosLivros.bibtexGoogleScholar/bibtexIndex.bib} % bib file is slsl.bib
%\bibliography{~/home/ricardo/Dropbox/35.ArtigosLivros.bibtexGoogleScholar/bibtexIndex.bib}
\bibliography{bibtexIndex.bib}
%\bibliography{slsl}


\vspace{5mm}
\begin{center}
\hspace{7mm}
\begin{tabular}{l}
   S\'ostenes L. Lins\\
   Centro de Inform\'atica, UFPE \\
   Av. Jornalista Anibal Fernandes s/n\\
   Recife, PE 50740-560 \\
   Brazil\\
   sostenes@cin.ufpe.br
\end{tabular}
\end{center}
